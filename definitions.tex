
\usepackage[utf8]{inputenc}
\usepackage{anyfontsize} % kill warnings about font sizes 
% \usepackage{amsmath}     % Mathematics FTW!
% \usepackage{amssymb}     % needed for \mathbb and mathcal
\usepackage{semantic}    % for PLT
\usepackage{stmaryrd}   % provides some PLy symbols, like Scott Brackets
\usepackage{mathpartir}  % for \mathpar
\usepackage{braket}      % for \set and \braket
\usepackage{thmtools}    % to duplicate theorems in the appendix
\usepackage{xspace}
\usepackage{proof}
\usepackage{myyhmath}
\usepackage{rotating}
\usepackage{tikz}
\usepackage{changepage}
\usepackage{listings}
\usepackage{mdframed}
\usepackage{mathtools}
\usepackage{ dsfont }
\usepackage[toc,page]{appendix}
\usepackage{pifont}
\usepackage[outline]{contour}
\usepackage{makecell}
\usepackage{stackengine,graphicx}
\usepackage{multicol}
% \usepackage[skip=0pt]{caption}
\usepackage{stackrel}
\usepackage{extarrows}
\usepackage{tabularx}
\usepackage{utfsym}
\usepackage{mdframed} % 用于添加边框
\usepackage[capitalise]{cleveref}
\usepackage{xcolor}
\usepackage{tikz}
\usetikzlibrary{matrix,positioning,arrows.meta}

\newcommand\xxoverset[3]{%
  \resizebox{#1+\widthof{\scriptsize #2}}{\height}{$#3$}}
\newcommand\extoverset[3][0pt]{%
  \mathrel{\overset{\textup{#2}}{\xxoverset{#1}{#2}{#3}}}}

\usetikzlibrary{shapes.geometric, arrows}

\tikzstyle{mynode} = [rectangle split, rectangle split parts=2, rounded corners, minimum width=3cm, minimum height=1cm,text centered, draw=black]
\tikzstyle{myarrow} = [->, >=stealth, double]
\tikzstyle{mytriangle} = [diamond, text centered, minimum width=3cm, minimum height=1.3cm, draw]


%%%%%%%%%%%%% Cleveref 
\crefname{section}{\textsection\!}{\textsection\!}
\crefname{figure}{Fig.}{Figs.}      % lowercsase: \cref
\Crefname{figure}{Figure}{Figures}  % uppercase: \Cref

%%%%%%%%%%%%%%%FOR GRAPH IMAGES
%\usepackage{tikz}
\usetikzlibrary{automata, graphs,positioning,chains,arrows,decorations.pathmorphing}


\usetikzlibrary{arrows,positioning,backgrounds} 



%\def\checkmark{\tikz\fill[scale=0.4](0,.35) -- (.25,0) -- (1,.7) -- (.25,.15) -- cycle;}

%%%%%%%%%%%%%%%


\newcommand{\cmark}{\ding{51}}%
\newcommand{\xmark}{\ding{55}}%

\newcommand{\lang}{\textsf{GTLC+}\xspace}%

\newcommand{\ie}{\emph{i.e.}\xspace}
\newcommand{\etal}{\emph{et al.}\xspace}
\newcommand{\eg}{\emph{e.g.}\xspace}
\newcommand{\resp}{\emph{resp.,}\xspace}

\usepackage{tabularx}

\declaretheorem[style=remark,numbered=no]{case}
\declaretheorem[style=remark,numbered=no]{notation}

\newtheorem{statement}{Statement}


% Comment form:  requires amssymb and color
\definecolor{violeta}{HTML}{9000FF}
\definecolor{propcolor}{HTML}{3F7D31}
\definecolor{etcolor}{HTML}{ED720E}

\newcommand{\node}[1]{\llparenthesis#1\rrparenthesis}
\newcommand{\mynote}[3]
   {{\color{#3} \fbox{\bfseries\sffamily\scriptsize#1}
   {\small$\blacktriangleright$\textsf{\emph{#2}}$\blacktriangleleft$}}~}
% \renewcommand{\mynote}[3]{}
\newcommand{\unsure}[1]{\textcolor{red}{#1}}

\definecolor{mynotecolor}{HTML}{F282B4}

\definecolor{customgreen}{HTML}{006400}

\newcommand{\et}[1]{\mynote{ET}{#1}{blue}}
\newcommand{\mt}[1]{\mynote{MT}{#1}{red}}
\newcommand{\jr}[1]{\mynote{JR}{#1}{violeta}}




\newcommand{\notes}[1]{\mynote{Note}{#1}{red}}

\newif\ifdiffmode
\diffmodefalse
\newcommand{\changed}[1]{\ifdiffmode\textcolor{blue}{#1}\else#1\fi}

\newenvironment{changedenv}
  {\begingroup\color{blue}}  % begin code -- Not using flag because there is a weird bug that I don't understand
  {\endgroup}

\renewenvironment{changedenv}
  {\begingroup} 
  {\endgroup}

\newcommand{\evcolor}[1]{{\color{blue} #1}}
\newcommand{\icolor}[1]{{\color{blue} #1}}

\newcommand{\?}{\textsf{\upshape ?}}
\newcommand{\consistent}[1]{\widetilde{#1}}
\newcommand{\collecting}[1]{\wideparen{#1}}
\newcommand{\cT}{G} %gradual type
\newcommand{\cE}{E} %gradual evidence type
\newcommand{\clT}{\collecting{T}}% collecting type
\newcommand{\ct}{\consistent{t}} %gradual term
\newcommand{\gvl}{\consistent{v}} %gradual value

% \newcommand{\Type}{\mathsf{Type}}
\newcommand{\Int}{\mathsf{Int}}
\newcommand{\String}{\mathsf{String}}
\newcommand{\Float}{\mathsf{Float}}
\newcommand{\Bool}{\mathsf{Bool}}
\newcommand{\ev}[1][]{\evcolor{\varepsilon_{#1}}}
\newcommand{\pr}[1]{\evcolor{\braket{{#1}}}}
\newcommand{\interior}[1]{\mathcal{I}_{#1}}
\newcommand{\meet}{\sqcap}
\newcommand{\false}{\mathsf{false}}

\newcommand{\D}{\mathcal{D}}
\newcommand{\E}{\mathcal{E}}
\newcommand{\emptyenv}{\ensuremath{\o}\xspace}
\newcommand{\trans}[1]{\mathbin{\evcolor{\circ^{#1}}}}
\newcommand{\red}{\longmapsto}
\newcommand{\nred}{~-->~}
\newcommand{\error}{\textup{\textbf{error}}}

\DeclareDocumentCommand{\itt}{ m O{} }{\icolor{\IfNoValueTF{#2}{t^{{#1}}}{t^{{#1}}_{#2}}}}
\DeclareDocumentCommand{\ittp}{ m O{} }{\icolor{\IfNoValueTF{#2}{t'^{{#1}}}{t'^{{#1}}_{#2}}}}
\newcommand{\TermT}[1]{\icolor{\mathds{T}}[#1]}
\newcommand{\ix}[1][]{\icolor{x^{{#1}}}}
\newcommand{\VarT}[1]{\mathds{V}[#1]}
\DeclareDocumentCommand{\itr}{ O{} }{\icolor{\IfNoValueTF{#1}{t}{t_{#1}}}}
\DeclareDocumentCommand{\itrp}{ O{} }{\icolor{\IfNoValueTF{#1}{t'}{t'_{#1}}}}
\newcommand{\iapp}[1]{\mathbin{\icolor{@^{#1}}}}
\newcommand{\braketeq}[2]{\evcolor{\braket{#1}}}
\newcommand{\iasc}[2]{\icolor{#1 :: #2}}

\newcommand{\relsymbol}{R}
\DeclareDocumentCommand{\rel}{m m}{\relsymbol(#1,#2)}
\DeclareDocumentCommand{\crel}{m m}{\widetilde{\relsymbol}(#1,#2)}
\newcommand{\dom}{\mathit{dom}}
\newcommand{\cod}{\mathit{cod}}


\newcommand{\boxit}[1]{\text{\lstinline{ref}}~{#1}}
\newcommand{\unbox}[1]{\text{\lstinline{!}}{#1}}
\newcommand{\assign}[2]{{#1} := {#2}}
\DeclareDocumentCommand{\vector}{m m}{\text{\lstinline{vec}}~{#1}~{#2}}
\DeclareDocumentCommand{\vectorset}{m m m}{\text{\lstinline{vset}}~{#1}~{#2}~{#3}}
\DeclareDocumentCommand{\vectorget}{m m}{\text{\lstinline{vget}}~{#1}~{#2}}
\DeclareDocumentCommand{\tuple}{m}{(#1)}
\DeclareDocumentCommand{\proj}{m m}{{#1}.(#2)}
\DeclareDocumentCommand{\letin}{m m m}{\text{\lstinline{let}}~{#1} = {#2}~\text{\lstinline{in}}~{#3}}
\DeclareDocumentCommand{\letrecin}{m m m}{\text{\lstinline{let rec}}~{#1} = {#2}~\textsf{in}~{#3}}
%\DeclareDocumentCommand{\ifthenelse}{m m m}{\textsf{if}~{#1}~\textsf{then}~{#2}~\textsf{else}~{#3}}
\DeclareDocumentCommand{\ifthenelse}{m m m}{\text{\lstinline{if}}~{#1}~\text{\lstinline{then}}~{#2}~\text{\lstinline{else}}~{#3}}
\DeclareDocumentCommand{\loopit}{m m m m}{\text{\lstinline{for}}~{#1}~=~{#2}~\text{\lstinline{to}}~{#3}~\text{\lstinline{do}}~{#4}~\text{\lstinline{done}}}
\DeclareDocumentCommand{\fun}{m m}{\text{\lstinline{fun}}~({#1}) -> {#2}}
%\DeclareDocumentCommand{\fun}{m m}{\textsf{fun}~({#1}) -> {#2} = e}
\DeclareDocumentCommand{\app}{m m}{{#1}~{#2}}
\DeclareDocumentCommand{\binop}{m m}{#1~\odot~#2}
\DeclareDocumentCommand{\datastruct}{m}{\text{\lstinline{type}}~{#1}~=~\evcolor{[}|~ C~\text{\lstinline{of}}~G \evcolor{]^{+}}}
\DeclareDocumentCommand{\match}{m}{\text{\lstinline{match}}~{#1}~\text{\lstinline{with}}~\evcolor{[}|~C~x\ldots \text{\lstinline{->}}~e\evcolor{]^{+}}}

\DeclareDocumentCommand{\ttuple}{m}{(#1)}
\DeclareDocumentCommand{\tvec}{m}{\text{\lstinline{vec}}[{#1}]}
\DeclareDocumentCommand{\tint}{}{\text{\lstinline{int}}}
\DeclareDocumentCommand{\tbool}{}{\text{\lstinline{bool}}}
\DeclareDocumentCommand{\tstring}{}{\text{\lstinline{string}}}
\DeclareDocumentCommand{\tfloat}{}{\text{\lstinline{float}}}
\DeclareDocumentCommand{\tunit}{}{\text{\lstinline{unit}}}
\DeclareDocumentCommand{\tunk}{}{\text{\lstinline{?}}}
\DeclareDocumentCommand{\tfun}{m m}{{#1} \to {#2}}
\DeclareDocumentCommand{\tref}{m}{\text{\lstinline{ref}}[{#1}]}

\DeclareDocumentCommand{\boxedtwo}{m m}{
  {
  \setlength{\fboxsep}{0pt}  % no padding between box border and contents
  \fbox{%
  \begin{tabular}{c|c}
    #1 & #2
  \end{tabular}%
  }
  }
}
\DeclareDocumentCommand{\boxedthree}{m m m}{
  {
  \setlength{\fboxsep}{0pt}  % no padding between box border and contents
  \fbox{%
  \begin{tabular}{c|c|c}
    #1 & #2 & #3
  \end{tabular}%
  }
  }
}


\newcommand{\true}{\mathtt{true}}

% OCaml style for listings
\lstdefinestyle{ocaml}{
  language=[Objective]Caml,      % built-in OCaml lexer
  basicstyle=\ttfamily\small,    % code font
  numbers=left,                  % line numbers (optional)
  numberstyle=\tiny,
  stepnumber=1,
  showstringspaces=false,
  columns=fullflexible,
  keepspaces=true,
  frame=single,
  framerule=0.3pt,
  breaklines=true,
  tabsize=2,
  keywordstyle={\bfseries\color{blue!70!black}},
  keywordstyle=[2]{\bfseries\color{teal!70!black}},
  commentstyle=\itshape\color{green!50!black},
  stringstyle=\color{purple!60!black},
  % Nice unicode and math replacements (optional)
  morekeywords={
    int,bool,float,char,string,unit,ref,mutable,fun,let rec,match,with,if,then,else,type,module,struct,open,include, vec, type, match, with, vget, vset, box, unbox,boxed, bflt, iflt, ifloat, bfloat, imm, box
  },
  morekeywords=[2]{dom,cod,ascribe,self},
  literate=
    {λ}{{$\lambda$}}1
    {->}{{$\to$}}2
    {=>}{{$\Rightarrow$}}2
    {>=}{{$\ge$}}2
    {<=}{{$\le$}}2
    {?}{{\bfseries\color{blue!70!black}?}}1
    {<}{{$\langle$}}1
    {>}{{$\rangle$}}1
}
\lstset{style=ocaml, mathescape=true}


\newcommand{\simdfo}{\stackrel{c}{=}} 
\newcommand{\csim}{\stackrel{c}{\sim}} % gradual

\newcommand{\BF}{\mathsf{BF}}
\newcommand{\FF}{\mathsf{FF}}
\newcommand{\FB}{\mathsf{FB}}
\newcommand{\BB}{\mathsf{BB}}

\newcommand{\Btype}{\text{\lstinline{bflt}}}
\newcommand{\Ftype}{\text{\lstinline{iflt}}}

\newcommand{\oblset}[1]{\textsc{#1}}
\newcommand{\GType}{\oblset{GType}}
\newcommand{\Type}{\oblset{Type}}
\newcommand{\Pow}{\ensuremath{\mathcal{P}}}
\newcommand{\RType}{\oblset{RType}}
\newcommand{\EType}{\oblset{EType}}

\newcommand{\R}{R}
\newcommand{\Rel}{\ensuremath{\mathcal{R}}}
\newcommand{\iev}[1][\Rel]{\mathcal{I}_{#1}}

\setlength{\parindent}{2em}
