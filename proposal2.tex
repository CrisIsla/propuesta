\documentclass[submission]{eptcs}
\setlength{\parindent}{0pt}
\providecommand{\event}{} % Name of the event you are submitting to
\usepackage{listings}
\usepackage[utf8]{inputenc}

\lstdefinelanguage{Lambda}{%
  morekeywords={%
    if,then,else,fix,fun,let % keywords go here
  },%
  morekeywords={[2]int, bool},   % types go here
  otherkeywords={:}, % operators go here
  literate={% replace strings with symbols
    {->}{{$\to$}}{2}
    {lambda}{{$\lambda$}}{1}
  },
  basicstyle={\ttfamily},
  keywordstyle={\bfseries},
  keywordstyle={[2]\itshape}, % style for types
  keepspaces,
  mathescape % optional
}[keywords,comments,strings]%
\setlength{\parindent}{2em}   % choose your indent width

\title{Optimizing GrEv}
\author{ 
\institute{Department of Computer Science\\
University of Chile\\
Santiago, Chile}
}
\begin{document}
\maketitle

\section{Introduction}

% background: existencia de grev, que quiero poder evaluar de mejor manera y optimizar

In the world of software engineering we often find ourselves in the dilema of choosing between two distinct tools: a flexible dynamic typed language, or a reliable staticly typed one.
On one side, dynamicaly typed languages like Python and JavaScript prioritize developer flexibility, rapid prototyping and expressiveness by deferring type checking to runtime.
On the other side, statically typed languages like OCaml and Rust prioritize early error detection, correctness and performance through ahead-of-time type verification.
Gradual typing has emerged as the principal option to reconcile this opposing paradigms, allowing a single language to support both disciplines seamlessly.
By enabling developers to control the presicion of static type checking at a fine-grained level (down to individual function arguments or variable bindings) gradual typing promises to 
deliver the best of both worlds, facilitating the evolution of scripts into robust applications without necessitating a complite rewrite in a different language.

However, the practical realization of gradual typing comes with significant engineering challenges.
The core difficulty lies in enforcing soundness without having considerable performance overheads.

% However, the realization of this promise is fundamentally constrained by the "performance cliff".
% Sound gradual typing requires that the runtime system enforce the invariants assumed by statically typed code components when they interact with dynamically typed (or less precisely typed) components.
% If this checks are inefficient, the overhead of mixing types can result in performance degradation so severe that it renders the gradual language impractical for production use.
% Consequently, the central research challenge in this field is the development of compilation techniques and runtime memory representations that minimize this interoperability overhead.

To date, the Grift compiler stands as the reference implementation for high-performance, close-to-the-metal gradual typing.
Grift employs a coercion-based approach, where type consistency checks are reified as "coercions" (specialized data structures representing the path between source and target types).

An alternative, and theoretically distinct, approach is Abstracting Gradual Typing (AGT).
AGT re-imagines gradual typing not as a translation to a cast calculus, but through the lens of Abstract Interpretation.
This model uses evidence; a runtime value that serves as justification for plausibility of a consistent typing judgment, to check for types.

The GrEv compiler represents the first attempt to build a high-performance compiler based on evidence.
The compiler has different semantic modes that change the way the language interacts with evidence.
This modes are GrEv/G (proxy, supports "normal" gradual typing semantics), GrEv/MC (monotonic closures) and GrEv/MV (monotonic values).
Preliminary evaluations indicate that GrEv is not only competitive with Grift but outperforms it in specific scenarios.
Also, results show that GrEv is more stable across configurations on the static-to-dynamic spectrum.
Despite it's great results, there still is a lot of room for improvement, and one of the biggest overheads that GrEv has is its evidence memory representation.

This thesis proposal aims to advance the state of the art of evidence-based gradual typing by exploring and implementing optimizations on memory representation of evidence on GrEv.

\section{Related Work}

To understand the necessity of the proposed research, one must first analyze the mechanisms employed by the current state-of-the-art compilers, Grift and GrEv.

\subsection{Grift: Coercions and Space Efficiency}

Grift serves as the reference implementation for coercion-based gradual typing. 
In this model, the static semantics of the gradual language are elaborated into an intermediate cast calculusm and this casts are then compiled into coercions.
One of the main challenges in this systems is that coercions can be accumulated.
For example, consider a function that is passed from a typed module to an untyped module and back again multiple times.
In a naive implementation, each boundary crossing adds a new wrapper to the function, wasting a lot of memory.
Grift mitigates this problem via coercion composition.
Whenever a new coercion is applied to an already coerted value, the runtime system attempts to compose the new coercion with the existing one.
If the coercions are inverses (casting \textit{Int} to \textit{?} and then back to \textit{Int}, for example), they cancel each other out, and the wrapper is removed.
If they are compatible, they are merged into a canonical form.
This normalization process ensures that the space overhead of wrappers remains bounded.

Grift also presents an optimization on access to mutable state using monotonic references.
In a traditional "guarded" approach, a mutable reference is wrapped in a proxy whenever it crosses a type boundary.
This allows different parts of the program to have different, potentially incompatible views of the same heap cell.
However, this requires every read and write operation to go through the proxy to get to the value.

Grift's monotonic mode (GriftMS) makes it so that every time a reference is cast to a more precise type, the original heap cell is updated.
If the cell contains a structure, it is recursively traversed to ensure all reachable values conform to the new type.

\subsection{GrEv: Evidence-based Compilation}

GrEv represents a paradigm shift, implementing the runtime semantics of AGT directly.
Instead of translating types into coercions, GrEv preserves the structure of gradual types at runtime in the form of evidence.
To better understand what evidence is, we first need to introduce some concepts:

% \begin{figure}[t]
%   \begin{displaymath}
%     \begin{array}{r@{\hspace{0.3em}}c@{\hspace{0.8em}}l@{\hspace{0.8em}}l}
%       e & ::= & \fun{\overline{x: \cT}}{e} \mid e~e\ldots \mid x  & \text{(functions and variables)} \\
%       & &  \mid n \mid f \mid \binop{e}{e} \mid \mathit{op}~e\ldots & \text{(integers, floats, and operators)} \\
%       & &  \mid b \mid \ifthenelse{e}{e}{e}  & \text{(booleans and conditionals)} \\
%       & &  \mid \loopit{x}{e}{e}{e} & \text{(loops)}\\
%       & &  \mid\letin{x: \cT}{e}{e} \mid \letrecin{f: \cT}{e}{e} & \text{(let bindings)} \\
%       & &  \mid\boxit{e} \mid \unbox{e} \mid \assign{e}{e} & \text{(references, unit)}\\
%       &  &   \mid\vector{e}{e} \mid \vectorset{e}{e}{e} \mid \vectorget{e}{e}& \text{(vectors)} \\
%       &  &  \mid\tuple{\overline{e}} \mid \proj{e}{e} & \text{(tuples)} \\
%       & &  \mid \datastruct{N} \mid C~e\ldots  & \text{(variant definition and construction)} \\
%       & &  \mid \match{e} & \text{(variant elimination)} \\
%       & & \mid e :: \cT \mid e ; e& \text{(ascriptions and sequences)} \\
%       \cT & ::= & \tint \mid \tbool \mid \tfloat \mid \tref{\cT} \mid \tvec{\cT} 
%       & \text{(types)} \\ 
%       %\ttuple{\overline{\cT}}
%       & & \mid \tunit \mid \cT * \cT \ldots \mid \tfun{(\overline{\cT})}{\cT} \mid N \mid \tunk  &  
%     \end{array}
%   \end{displaymath}
%   \caption{\lang source syntax}
%   % \mt{what is the second argument of vector? Are recursive types missing?}
%   %\et{box/unbox/setbox or ml-style !, :=, ref}\et{finalize keywords and update lstlisting}
%   % \et{fix syntax operators, list them}\et{what about unit?}}
%   \label{fig:lang-syntax}
% \end{figure} 

\subsubsection{Concretization and Abstraction}

Gradual types can be understood as sets of static types. For example, the unknown type $?$ represents every static type.

The concretization function ($\gamma$) maps a gradual type to the set of static types it represents.
For precise types (like int or bool), the mapping is singular:

$$\gamma(int) = {int}$$

and, as we said before, the unknown type $?$ concretizes to the set of all possible static types in the language:

$$\gamma(?) = TYPE$$

Structural types propagate this relation. A gradual function type $G_1 \rightarrow G_2$ concretizes to the set of all static function types $T_1 \rightarrow T_2$ where $T_1$ is the concretization of $G_1$ and $T_2$ is the concretization of $G_2$.

With this definition we can think about \textbf{presicion}.
A gradual type $G_1$ is less precise than $G_2$ if the set of static types represented by $G_1$ (i.e. its concretization) is a superset of those represented by $G_2$.

The abstraction function ($\alpha$) maps a set of static types to the more precise gradual type that represents the set.
For example:

$$\alpha(\{int, bool\}) = ?$$

or

$$\alpha(\{int \rightarrow bool, int \rightarrow string\}) = int \rightarrow ?$$

\subsubsection{Consistency}

In a static system there could be different type relations, like equality, subtyping, containtment, etc.
What AGT proposes is that, in order to get the statics semantics of a gradual system, this relations must be lifted into their gradual counterparts.

The gradual counter part of equality is called consistency (denoted by $\sim$).
We say that two gradual types $G_1$ and $G_2$ are consistent if there exist a type $T$ that belong to the concretization of both $G_1$ and $G_2$.

$$G_1 \sim G_2 \Leftrightarrow \gamma (G_1) \cap \gamma (G_2) \ne \emptyset$$

This explains why $int$ is consistent with $?$ ($int$ is in the set of all types), but not consistent with $bool$.
A very important detail is that, different from equality, \textbf{consistency is not transitive}.
If it were, we could have something like the following:

$$int\sim ? \land ? \sim bool \Rightarrow int \sim bool$$

which breaks everything.

\subsubsection{Evidence and Consistent Transitivity}

AGT provides a direct dynamic semantics of gradual programs by applying proof reduction on gradual typing derivations.

While in a fully static language equality between types is transitive, in a gradual setting we can not statically make sure that the transitivity of consistency between types is valid.
Lets introduce the following program:

\begin{lstlisting}[language=Lambda]
let f: int -> int = fun x -> x + x in 
  f (true :: ?)
\end{lstlisting}

In this example, the function \textbf{\textit{f}} expects an \textit{int} and a value of type \textit{?} is given.
This program type checks because $? \sim int$, but in runtime, as the argument of the function given is a boolean, we have a chain:

$$bool \sim ? \land ? \sim int$$

So dynamically something must be done to justify this judgement.

Evidence is a runtime object that carries information about the type of some value.
This information is then used to justify the judgement.
When two values are met in an operation?, their evidence is combined to check 
In the example transitivity does not work, because $int \nsim bool$, and this is something that the dynamic system should be able to deduce.

EXPLICACIÓN DE EVIDENCIA, Y CONSISTENT TRANSITIVITY

When more complex types and relations are available (like subtyping, for example), evidence needs to be able to store the information about the types involved and the relations between them.
GrEv only has simple types, so evidence can be represented as a single type.









% We have a function \textit{double} with type $?$ that expects its parameter to be an $int$ (we know that because its definition has the $+$ operator, that is only valid with $int$), but we call it with the value \textit{true} of type $bool$.
% We know statically that $bool \sim ?$ and $? \sim int$, but in this case the transitivity is not valid because \textit{bool} is not consistent with \textit{int}.
%
% To solve this problem, AGT proposes the use of \textit{evidence} to make sure that the transitivity between this consistenty judgements holds.
% Evidence is a runtime object that justifies a specific consistency judgement. 
% As a value flows from a context to another (like in the example), evidence is checked to see if the types are compatible, and combined into a more precise state.
% This operation is called \textit{Consistent Transitivity}.

\subsubsection{GrEv's evidence implementation}

% Apart from GrEv's gradual modes, GrEv also has it's static and dynamic versions (StaticGrEv and DynGrEv, respectively).
% This versions follow the basic principles of compilation for statically and dynamically typed languages.
As mentioned before, the current implementation of GrEv only supports simple types, so evidence can be represented as a simple type.
GrEv differentiate between inmmediate and boxed values.
Boxed values have an specific field in memory that saves its evidence.
For example, a reference is stored as two words (16 bytes), one as the evidence and the other as the value.
Inmmediate values are tagged, so there is no need to have extra memory to get it's evidence.

\subsection{GrEv modes}

As stated in the introduction, GrEv supports different semantic modes that interacts with evidence in different ways.
To ilustrate the semantic each of the modes supports, let's introduce the following program:

\begin{lstlisting}[language=Lambda]
let double: ? = fun x -> x + x in 
  foo 1;
  foo 1.0
\end{lstlisting}

\subsubsection{GrEv/MC: Monotonic closures}

GrEv/MC implements monotonic closures, this is, the evidence for a closure is global.
This means that, when the evidence of the closure is updated, this affects every other application of the function.
In the example, when \textit{foo} is applied to 1, its evidence is updated from \textit{?} to $int \rightarrow int$, thus its following application throws an error.


\subsubsection{GrEv/G: Guarded}

GrEv/G uses proxys to store values in the heap separated from their evidence.
In this version, every time  the value of an evidence is updated, a shallow copy is created with a new memory cell that contains this update.
In this case, after the first application of \textit{foo}, a new memory cell is created that contains the evidence $int \rightarrow int$, so the second application throws no error.


\subsubsection{GrEv/MV: Monotonic values}

GrEv/MV uses monotonic evidences for every value in the language.
In this sense, when the evidence of a reference, a vector, a closure, etc. is updated, it is updated globally and affects the following uses of that value.

\section{Problem}


\section{Research questions}

The questions regarding this thesis are the following:

\section{Hypothesis}

\section{Objectives}

\section{Methodology}

\section{Contributions}

\section{Conclusion}

% \section{Main Goal}
%
%
% \section{Specific goals}
%
% \begin{itemize}
% \item goal 1
% \end{itemize}
%
% \section{Methodology}
%
% texto
%
% \section{Expected results}
%
%  y contribuciones

\nocite{*}
\bibliographystyle{eptcs}
\bibliography{bibliography}
\end{document}
